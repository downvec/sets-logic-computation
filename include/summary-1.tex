A \emph{set} is a collection of objects, the elements of the set. We
write $x \in X$ if $x$~is an element of~$X$. Sets are
\emph{extensional}---they are completely determined by their elements.
Sets are specified by \emph{listing} the elements explicitly or by
giving a property the elements share
(\emph{abstraction}). Extensionality means that the order or way of
listing or specifying the elements of a set doesn't matter. To prove
that $X$ and~$Y$ are the same set ($X = Y$) one has to prove that
every element of~$X$ is an element of~$Y$ and vice versa.

Important sets include the natural ($\Nat$), integer ($\Int$), rational
($\Rat$), and real ($\Real$) numbers, but also \emph{strings} ($X^*$)
and infinite \emph{sequences} ($X^\omega$) of objects. $X$ is a
\emph{subset} of $Y$, $X \subseteq Y$, if every element of $X$ is also
one of~$Y$. The collection of all subsets of a set~$Y$ is itself a
set, the \emph{power set} $\Pow{Y}$ of~$Y$. We can form the
\emph{union} $X \cup Y$ and \emph{intersection} $X \cap Y$ of sets. An
\emph{ordered pair} $\tuple{x, y}$ consists of two objects $x$ and~$y$, but in
that specific order. The pairs $\tuple{x, y}$ and $\tuple{y, x}$ are
different pairs (unless $x = y$). The set of all pairs $\tuple{x, y}$
where $x \in X$ and $y\in Y$ is called the \emph{Cartesian product}~$X
\times Y$ of $X$ and $Y$. We write $X^2$ for $X \times X$; so for
instance $\Nat^2$ is the set of pairs of natural numbers.
