Turing machines are determined by their instruction sets, which are
finite sets of quintuples (for every state and symbol read, specify
new state, symbol written, and movement of the head). The finite sets
of quintuples are enumerable, so there is a way of associating a
number with each Turing machine instruction set. The \emph{index} of
a Turing machine is the number associated with its instruction set
under a fixed such schema. In this way we can ``talk about'' Turing
machines indirectly---by talking about their indices.

One important problem about the behavior of Turing machines is whether
they eventually halt. Let $h(e, n)$ be the function which $= 1$ if the
Turing machine with index~$e$ halts when started on input~$n$, and
$=0$ otherwise. It is called the \emph{halting function}. The question
of whether the halting function is itself Turing computable is called
the \emph{halting problem}. The answer is no: the halting problem is
unsolvable. This is established using a diagonal argument.

The halting problem is only one example of a larger class of problems
of the form ``can $X$ be accomplished using Turing machines.'' Another
central problem of logic is the \emph{decision problem for first-order
  logic}: is there a Turing machine that can decide if a given
sentence is valid or not. This famous problem was also solved
negatively: the decision problem is unsolvable. This is established by
a reduction argument: we can associate with each Turing machine~$M$
and input~$w$ a first-order sentence~$T(M, w) \lif E(M, w)$ which is
valid iff $M$ halts when started on input~$w$. If the decision problem
were solvable, we could thus use it to solve the halting problem.
