The \emph{completeness theorem} is the converse of the \emph{soundness
  theorem}. In one form it states that if $\Gamma \Entails !A$ then
$\Gamma \Proves !A$, in another that if $\Gamma$ is consistent then it
is satisfiable. We proved the second form (and derived the first from
the second). The proof is involved and requires a number of steps. We
start with a consistent set~$\Gamma$. First we add infinitely many
new constant symbols~$c_i$ as well as formulas of the form
$\lexists[x][!A(x)] \lif !A(c)$ where each formula~$!A(x)$ with a free
variable in the expanded language is paired with one of the new
constants. This results in a \emph{saturated} consistent set of
sentences containing~$\Gamma$. It is still consistent. Now we take
that set and extend it to a \emph{maximally consistent set}. A
maximally consistent set has the nice property that for any
sentence~$!A$, either $!A$ or $\lnot !A$ is in the set. Since we started
from a saturated set, we now have a saturated and maximally consistent
set of sentences~$\Gamma^*$ that includes~$\Gamma$. From this set it
is now possible to define a structure~$\Struct{M}$ such that
$\Sat{M(\Gamma^*)}{!A}$ iff $!A \in \Gamma^*$. In particular,
$\Sat{M(\Gamma^*)}{\Gamma}$, i.e., $\Gamma$ is satisfiable. If $=$ is
present, the construction is slightly more complex.

Two important corollaries follow from the completeness theorem. The
\emph{compactness theorem} states that $\Gamma \Entails !A$ iff
$\Gamma_0 \Entails !A$ for some finite $\Gamma_0 \subseteq \Gamma$. An
equivalent formulation is that $\Gamma$ is satisfiable iff every
finite $\Gamma_0 \subseteq \Gamma$ is satisfiable. The compactness
theorem is useful to prove the existence of structures with certain
properties. For instance, we can use it to show that there are
infinite models for every theory which has arbitrarily large finite
models. This means in particular that finitude cannot be expressed in
first-order logic. The second corollary, the \emph{L\"owenheim-Skolem
  Theorem}, states that every satisfiable $\Gamma$ has a countable
model. It in turn shows that uncountability cannot be expressed in
first-order logic.
