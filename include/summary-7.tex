\emph{Proof systems} provide purely syntactic methods for
characterizing consequence and compatibility between sentences.
\iftag{prfND}{\emph{Natural deduction}}{The \emph{sequent calculus}}
is one such proof system. A \emph{derivation} in it consists of a tree
of \iftag{prfND}{formulas}{sequents (a sequent $\Gamma \Sequent
  \Delta$ consists of two sequences of formulas separated
  by~$\Sequent$)}. The topmost \iftag{prfND}{formulas}{sequents} in a
derivation are \iftag{prfND}{\emph{assumptions}}{\emph{initial
    sequents} of the form $!A \Sequent !A$}.  All other
\iftag{prfND}{formulas}{sequents}, for the derivation to be correct,
must be correctly justified by one of a number of \emph{inference
  rules}. These come in pairs; \iftag{prfND}{an introduction and an
  elimination rule }{a rule for operating on the left and on the right
  side of a sequent} for each connective and quantifier. For instance,
\iftag{prfND}{if a formula~$!A$ is justified by a $\Elim{\lif}$ rule,
  the preceding formulas (the \emph{premises}) must be $!B \lif !A$
  and $!B$ (for some~$!B$). Some inference rules also allow
  assumptions to be \emph{discharged}. For instance, if $!A \lif !B$
  is inferred from $!B$ using $\Intro{\lif}$, any occurrences of~$!A$
  as assumptions in the derivation leading to the premise~$!B$ may be
  discharged, given a label that is also recorded at the
  inference.}{if a sequent $\Gamma \Sequent \Delta, !A \lif !B$ is
  justified by the $\RightR{\lif}$ rule, the preceding sequent (the
  \emph{premise}) must be $!A, \Gamma \Sequent \Delta, !B$. Some rules
  also allow the order or number of sentences in a sequent to be
  manipulated, e.g., the \RightR{\Exchange} rule allows two formulas
  on the right side of a sequent to be switched.}

If there is a derivation \iftag{prfND}{with end formula~$!A$ and all
  assumptions are discharged}{of the sequent $\quad \Sequent !A$}, we
say $!A$ is a \emph{theorem} and write~$\Proves !A$. If
\iftag{prfND}{all undischarged assumptions are in some
  set~$\Gamma$}{there is a derivation of $\Gamma_0 \Sequent !A$ where
  every $!B$ in $\Gamma_0$ is in~$\Gamma$}, we say $!A$ is
\emph{derivable from}~$\Gamma$ and write $\Gamma \Proves !A$. If
\iftag{prfND}{$\Gamma \Proves \lfalse$}{there is a derivation of
  $\Gamma_0 \Sequent \quad$ where every $!B$ in $\Gamma_0$ is
  in~$\Gamma$} we say $\Gamma$ is \emph{inconsistent}, otherwise
\emph{consistent}. These notions are interrelated, e.g., $\Gamma
\Proves !A$ iff $\Gamma \cup \{\lnot !A\}$ is inconsistent. They are
also related to the corresponding semantic notions, e.g., if $\Gamma
\Proves !A$ then $\Gamma \Entails !A$. This property of proof
systems---what can be derived from $\Gamma$ is guaranteed to be
entailed by~$\Gamma$---is called \emph{soundness}. The \emph{soundness
  theorem} is proved by induction on the length of derivations,
showing that each individual inference preserves
\iftag{prfND}{entailment of its conclusion from open assumptions
  provided its premises are entailed by their open
  assumptions}{validity of the conclusion sequent provided the premise
  sequents are valid}.
